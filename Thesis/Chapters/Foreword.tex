\chapter{Foreword}

\label{Foreword}

Minecraft is one of, if not the most popular modern game on the market. Loved and known by millions of people, it offers the player a vast, near endless world and the creative freedom to build whatever their imagination allows. The game continues to add new features and blocks to its inventory, allowing for ever-expanding horizons of potential for each one of the players. The vast possibilities of gameplay allow for many play-styles to occur. There are the standard survival players that play the game by the game's rules, slowly building up their base and building small farms, exactly as the creators of the game intended. Then there are the builders, the artists of Minecraft, who express their creativity within the scope of the game and its array of blocks and construct awe-inspiring builds on every scale, from small houses with incredible detail in their design to cities, castles or statues that scratch the ceiling of the world. There are the players that add their own modifications on to the foundation that Minecraft provides, to the extent that it almost seems like another game entirely. Each player can choose what they want to do, and is never confined to just one category, since there are usually no clear lines distinguishing one play-style from another.

Myself, I chose to explore the area of computational redstone, after being blown away by the things that the community has created. Seeing actual computers and calculators in this incredible game left their mark in me, and while the builders of Minecraft seemed just as alluring, computational redstone was more intriguing. With some experience in basic programming and having created some very simple games that, admittedly, are not worthy of mention here, my interests began wandering deeper into the functionality of the computers we interact with almost every moment of our waking lives. As with most subjects nowadays, this one was best explored online, specifically YouTube, where there is an enormous amount of information regarding computer science and computational logic. I'd like to point out two people in particular who have by far taught me the most in this subject. Ben Eater \cite{BenEater} had previously worked at Khan Academy, one of the best online learning sources. Eater creates educational YouTube videos on all things regarding the inner workings of computers and procedures on the lowest levels, working on breadboards to create his own circuits. His most impressive series is the 8-bit CPU on a breadboard, where he goes into the details of why and how everything he makes is needed and works, and which served as the main inspiration for this thesis. The other person is Mattbatwings, another educational video creator, is a computer science student who focuses on teaching computational logic in Minecraft to a broad audience. He laid the foundations for many concepts and circuits which I've modified and built upon in creating my CPU.

Mods, or modifications, are a big part in designing larger creations, both in the realm of redstone or that of builders. The big one is \textit{World Edit}, allows for copying and pasting volumes, or stacking volumes in a series of itself, a feature that, when looking at the RAM, the big chunk of memory behind the CPU, can quickly be justified for its use. This mod is an incredible time saver, allowing for rather quick iteration, without which designing this CPU within the span of a year would have been practically impossible. The other mod, which came into play for debugging and executing programs is the \textit{Carpet} mod for its time-control features, where the speed of the game can be modified from 20x slower up to 50x faster.

Designing the CPU was largely a case of trial and error. The knowledge gained over the past few years served almost entirely as a foundation upon which the layout, architecture and interaction between each component were designed. Initially a prototype was made to better understand the relationships between each part of the CPU and how the instructions could work. Many things changed between the prototype and the final design, improving the connections to be more efficient, but also carrying over some design flaws and inefficiencies, which will be discussed later on.